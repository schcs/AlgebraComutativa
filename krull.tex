\documentclass[12pt]{amsart}
\usepackage{amsthm}
\usepackage{amssymb}
\usepackage{showkeys}
\usepackage{tikz-cd}

\renewcommand{\a}{\mathfrak a}
\renewcommand{\b}{\mathfrak b}
\newcommand{\m}{\mathfrak m}
\newcommand{\n}{\mathfrak n}
\newcommand{\p}{\mathfrak p}
\newcommand{\q}{\mathfrak q}
\renewcommand{\r}{\mathfrak r}
\newcommand{\F}{\mathbb F}
\renewcommand{\L}{\mathbb L}
\newcommand{\Q}{\mathbb Q}
\newcommand{\N}{\mathbb N}
\newcommand{\Z}{\mathbb Z}


\newcommand{\K}{\mathbb K}
\newcommand{\fracf}[1]{\mbox{Frac}(#1)}
\newcommand{\spec}[1]{\mbox{Spec}(#1)}
\newcommand{\len}{\mbox{len}\,}

\newtheorem{theorem}{Theorem}
\newtheorem{corollary}{Corollary}[theorem]
\newtheorem{lemma}[theorem]{Lemma}
\newtheorem{exercise}[theorem]{Exercício}

\theoremstyle{definition}
\newtheorem{example}[theorem]{Exemplo}


\oddsidemargin 0pt
\evensidemargin 0pt
\textheight 8.1in \textwidth 6.3in


\relpenalty=10000
\binoppenalty=10000
\tolerance=500


\begin{document}

%\Large
\title{O Teorema de dimensão de Krull}
\maketitle

\begin{lemma}
    Seja $(R,\m,k)$ um anel local notheriano. Seja $a\in\m$ e seja $S=R/(a)$. Então 
    \begin{enumerate}
        \item $\deg\lambda_S\in\{\deg\lambda_A-1,\deg\lambda_A\}$;
        \item Se $a\in R$ não é divisor de zero, então $\deg\lambda_B=\deg \lambda_A-1$. 
    \end{enumerate}
\end{lemma}
\begin{proof}
    Seja $\bar\m=\bar\m/(a)$ a imagem de $\m$ em $S$. O anel $S$ é anel local noetheriano com ideal maximal 
    $\bar m$ com o mesmo corpo residual, pois $(R/(a))/(\m/(a))\cong A/\m=k$.  
\end{proof}
Temos que 
\[
    S/\bar \m^n\cong R/((a)\cap \m^n)\quad\mbox{e}\quad ((a)+\m^n)/\m^n\cong (a)/((a)\cap \m^n).
\]
Portanto temos uma seqência exata
\[
    0\to (a)/((a)+\m^n)\to R/\m^n\to S/\bar\m^n\to 0.
\]
Portanto 
\[
    \len(R/\m^n)+\len(S/\bar\m^n)+\len((a)/((a)\cap \m^n));
\]
ou seja 
\[
    \lambda_R(n)=\lambda_S(n)+\len_R((a)/((a)+\m^n)).
\]
Em particular, $\lambda_S(n)\leq \lambda_R(n)$ e $\deg\lambda_S\leq\deg\lambda_R$. 

A existência do mapa sobrejetor
\[
    R/\m^{n-1}\to (a)/((a)\cap \m^n)
\]
induzido pela multiplicação por $a$ implica par todo $n\geq 1$ que 
\[
    \len((a)/((a)\cap\m^n))\geq\len R/\m^{n-1}=\lambda_A(n-1).
\]
Portanto, 
\[
    \lambda_S(n)\geq \lambda_R(n)-\lambda_R(n-1)
\]
e 
\[
    \deg\lambda_S\geq \deg\lambda_R-1.
\]
Assuma que $a$ não é divisor de zero. Aplicando o Teorema de Artin--Rees com $M=R$, $N=(a)$ e $\a=\m$, existe 
$c>0$ tal que $(a)\cap \m^n\subseteq (a)\m^{m-c}$ para todo $n$ suficientemente grande. 
Assim temos 
\[
    (a)/((a)\cap \m^n)\to (a)/((a) \m^{n-c})\cong R/\m^{n-c}.
\]
O último isomorfismo está induzido pela multiplicação por $a$ usando que $a$ não é divisor de zero. 
Assim, temos que 
\[
    \lambda_S(n)\leq\lambda_R(n)-\lambda_R(n-c)
\]
e 
\[
    \deg\lambda_S\leq\deg\lambda_R-1.
\]
\end{document}
