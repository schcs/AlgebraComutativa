\documentclass[12pt]{amsart}
\usepackage{amsthm}
\usepackage{amssymb}
\usepackage{showkeys}
\usepackage{tikz-cd}

\renewcommand{\a}{\mathfrak a}
\renewcommand{\b}{\mathfrak b}
\newcommand{\m}{\mathfrak m}
\newcommand{\n}{\mathfrak n}
\newcommand{\p}{\mathfrak p}
\newcommand{\q}{\mathfrak q}
\renewcommand{\r}{\mathfrak r}
\newcommand{\F}{\mathbb F}
\renewcommand{\L}{\mathbb L}
\newcommand{\Q}{\mathbb Q}
\newcommand{\N}{\mathbb N}
\newcommand{\Z}{\mathbb Z}


\newcommand{\K}{\mathbb K}
\newcommand{\fracf}[1]{\mbox{Frac}(#1)}
\newcommand{\spec}[1]{\mbox{Spec}(#1)}
\newcommand{\len}{\mbox{len}\,}

\newtheorem{theorem}{Theorem}
\newtheorem{corollary}{Corollary}[theorem]
\newtheorem{lemma}[theorem]{Lemma}
\newtheorem{exercise}[theorem]{Exercício}

\theoremstyle{definition}
\newtheorem{example}[theorem]{Exemplo}


\oddsidemargin 0pt
\evensidemargin 0pt
\textheight 8.1in \textwidth 6.3in


\relpenalty=10000
\binoppenalty=10000
\tolerance=500


\begin{document}

%\Large
\title{Funções de Hilbert}
\maketitle
\section{Comprimento}

Seja $M\neq 0$ um $R$-módulo. $M$ é dito \emph{simples} (ou \emph{irredutível}), se $0$ e $M$ são os únicos $R$-submódulos de $M$. 
Uma cadeia de submódulos 
\[
    0\subset M_1\subset M_2\subset \cdots \subset M_r=M
\]
é dito série de composição de comprimento $r$ para $M$ se $M_i/M_{i+1}$ são simples. 
O comprimento $\len M$ de $M$ é definida como o 
mínimo entre os comprimentos de séries de composição de $M$. O comprimento de $M$ é infinito se $M$ não possui 
séries de composição finita. 

\begin{example}
    Se $k$ é um corpo, então $k$-módulos são espaços vetoriais, e $\len V=\dim V$ para todo $k$-espaço $V$.
\end{example}

\begin{lemma}
    $M$ é simples se e somente se $M\cong R/\m$ com algum ideal maximal $\m$ de $R$.
\end{lemma}
\begin{proof}
    Se $\m$ é maximal, então $R/\m$ é simples, pelo Teorema de Correspondência. 
    Seja $M$ simples. Existe $m\in M$ tal que $Rm\neq 0$. Como $Rm$ é um submódulo de $M$, temos que 
    $Rm=M$.  Defina 
    $\psi: R\to M$, por $r\mapsto rm$. Então $\psi$ é sobrejetiva, e $M\cong R/\ker\psi$. Pela simplicidade 
    de $M$, temos que $\ker\psi$ é maximal.
\end{proof}


\begin{theorem}
    As seguintes propriedades são válidas.
\begin{enumerate}
    \item $\len M$ é finita se e somente se $M$ é noetheriano e artiniano. 
    \item Se $\len M$ é finita, então toda série de composição tem  comprimento $\len M$.  
    \item Se
    \[
        0\to N\to M\to P\to 0
    \]
    é uma sequência exata de $R$-módulos, então $\len M=\len N+\len P$. 
\end{enumerate}
\end{theorem}
\begin{proof}
    (1) Exercício.

    (2) Indução por $\len M$. Se $\len M=0$, então $M=0$ e o teorema está trivialmente válido. 
    Assuma que 
    \[
        0\subset M_1\subset M_2\subset \cdots \subset M_r=M
    \]
    é uma série de composição para $M$. Então $\len M\leq r$. Além disso, $M_1$ é simples e 
    \[
0=M_1/M_1\subset M_2/M_1\subset\cdots \subset M_r/M_1=M/M_1
    \]
    é uma série de composição para $M/M_1$. Pela definição do comprimento, $\len M/M_1\leq r-1$ e 
    pela hipótese de indução $\len M/M_1=r-1$ e toda série de composição de $M/M_1$ tem comprimento 
    $r-1$. 

    Assuma que 
    \[
        0\subset N_1\subset\cdots\subset N_k=M.
    \]
    é uma série de composição para $M$. Assuma que $i$ é minimal tal que $N_i\cap M_1\neq 0$. Pela simplicidade 
    de $M_1$, temos que $N_i\cap M_1=M_1$; ou seja $M_1\subseteq N_i$. Então afirmamos que 
    \begin{equation}\label{eq:comp}
       0\subset (N_1+M_1)/M_1\subset\cdots\subset (N_{i-1}+M_1)/M_1\subseteq N_i/M_1 \subset\cdots
       \subset N_k/M_1=M/M_1
    \end{equation}
    é uma série de composição para $M/M_1$ com comprimento $r-1$ (ou seja a inclusão $\subseteq$ no meio é 
    $=$). Considerando um quociente para $j<i$, temos que 
    $N_j\cap M_1=N_{j-1}\cap M_1=0$ e assim $(N_j+M_1)/M_1\cong N_j/(N_j\cap M_1)=N_j$ e 
    $(N_{j+1}+M_1)/M_1\cong N_{j+1}/(N_{j+1}\cap M_1)=N_{j+1}$. Portanto
    \[%\begin{align*}
        ((N_{j}+M_1)/M_1)/((N_{j-1}+M_1)/M_1)\cong
        N_j/N_{j-1}
    \]%\end{align*}
    que é simples. 
    Se $j\geq i$, então 
    \[
        (N_{j+1}/M_1)/(N_{j}/M_1)\cong N_{j+1}/N_{j}
    \]
    é simples. 
    Finalmente 
    \[
        (N_i/M_1)/((N_{i-1}+ M_1)/M_1)\cong N_i/(N_{i-1}+M_1).
    \]
    Mas $N_{i-1}\subset N_{i-1}+M_1\subseteq N_{i}$. Como $N_i/N_{i-1}$ é simples, temos que 
    $N_{i-1}+M_1= N_i$. Ou seja, $N_i/(N_{i-1}+ M_1)=0$. Logo \eqref{eq:comp} é uma séria de 
    composição para $M/M_1$ de comprimento $r-1$. Assim $k-1=r-1$ e $r=k$. 

    (3) Exercício.
\end{proof}

\section{Polinômios binomiais}

Um polinômio binomial é um polinômio na forma 
$$
    \binom xd=\frac{x(x-1)\cdots(x-d+1)}{d!}
$$
onde $d\geq 0$. %Se $d<0$, então $\binom{x}{d}=0$.

\begin{lemma}
    (1) Seja $p(x)\in\mathbb Q[x]$ um polinômio de grau $d$. Temos que $p(n)\in\mathbb Z$ para todo $n\in\mathbb N$ 
suficientemente grande se e somente se 
$$
    p(x)=a_d \binom{x}{d}+a_{d-1}\binom{x}{d-1}+\cdots+a_0\binom{x}{0}.
$$
com $a_i\in\Z$. 

(2) Assuma que $f:\mathbb N\to\mathbb N$ é uma função. Suponha que existe um polinômio $q(t)\in\mathbb Q[t]$ de 
grau $k-1$ tal que 
$$
\Delta f(n)=f(n+1)-f(n)=q(n)
$$
para todo natural $n$ suficientemente grande. Então existe um polinômio $p(t)\in\mathbb Q[t]$ 
de grau $k$ tal que $f(n)=p(n)$ para todo natural $n$ suficientemente grande. 
\end{lemma}

\section{Anéis graduados}

Um anel $R$ é dito \emph{graduado} se 
$$
R=R_0\oplus R_1\oplus R_2\oplus \cdots
$$
como um grupo abeliano onde $R_iR_j\subseteq R_{i+j}$. Em particular, $R_0$ é um anel, $R_i$ é um $R_0$-módulo 
para todo $i\geq 0$ e $R$ é uma $R_0$-álgebra. Os $R_i$ são chamados de \emph{componentes homegêneos} de $R$ e um 
elemento $f\in R_i$ é dito \emph{homogêneo} de grau $i$. 

Seja $R$ um anel. Considere a cadeia $F_i$ de ideais 
\[
R = F_0\supset F_1\supset F_2\supset\cdots 
\]
tal que $F_iF_j\subseteq F_{i+j}$. Uma tal cadeia chama-se {\em filtração} sobre $R$. Dada uma filtração 
como na linha destacada anterior, podemos definir
\[
    R_{\rm gr}=\bigoplus_{i\geq 0}F_{i}/F_{i+1}
\]
como um grupo abeliano e um produto em $R_{\rm gr}$ pela regra
\[
    (x_i+F_{i+1})(x_j+F_{j+1})=x_ix_j+F_{i+j+1}
\]
para $x_i\in F_i$ e $x_j\in F_{j}$ e estender estes produtos linearmente para $R_{\rm gr}$.

Se $R$ é uma $S$-álgebra, então tomamos $S$-módulos $R_i$ na definição de graduação. Em particular, 
se $R$ é uma $k$-álgebra com um corpo $k$, então $R_i$ é um $k$-espaço vetorial.
\begin{example} 
    A álgebra $R=k[x_1,\ldots,x_n]$ de polinômios é uma $k$-álgebra graduada com a graduação na qual $R_i$ é 
o $k$-espaço de polinômios com grau (total) $i$. 
\end{example}

\begin{example} 
Um ideal $I\subseteq R$ de um anel graduado é dito \emph{homogêneo} se 
$$
I=I_0\oplus I_1\oplus I_2\oplus \cdots 
$$
onde $I_i=I\cap R_i$. Se $I$ é um ideal homogêneo, então 
$$
    R/I=\left(\bigoplus_{i\geq 0} R_i\right)/I=\bigoplus_{i\geq 0}((R_i+I)/I)\cong \bigoplus_{i\geq 0} (R_i/I_i).
$$
Se $r_i\in R_i/I_i\cong (R_i+I)/I$ e $r_j\in R_j/I_j\cong (R_j+I)/I$ então 
\[
    r_ir_j\in (R_iR_j+I)/I\subseteq (R_{i+j}+I)/I\cong R_{i+j}/I_{i+j}.
\] 
Ou seja, o quociente $R/I$ é graduado.
\end{example}

\begin{example} Seja $(R,\mathfrak m,k)$ um anel local noetheriano. Considere a filtração 
    \[
        R\supset\m\supset\m^2\supset\cdots.  
    \]
    Podemos definir
    \[
        R_{{\rm gr}}=k+\m/\m^2\oplus \m^2/\m^3\oplus \cdots
    \]
    com a multiplicação como acima.
    Então $\m^i/\m^{i+1}$ são $k$-espaços 
    vetoriais e $R_{{\rm gr}}$ é uma 
    soma direta de $k$-espaços vetoriais.
    Assim $R_{{\rm gr}}$ é uma $k$-álgebra graduada. Assuma que $m_1,\ldots,m_r\in m$ é um sistema gerador
    do ideal $\m$. Então $m_1+\m^2,\ldots,m_r+\m^2$ é um sistema gerador para $R_{{\rm gr}}$. Em particular, 
    $R_{\rm gr}$ é quociente do anel $k[x_1,\ldots,x_r]$ de polinômios de posto $r$.
\end{example}
    


\begin{exercise}
    Mostre que um ideal $I\subseteq k[x_1,\ldots,x_n]$ é homogêneo se e somente se $I$ é gerado por 
    $f_1,\ldots,f_k$ onde $f_i$ são polinômios homogêneos. 
\end{exercise}


Se $R$ é um anel graduado e $M$ é um $R$-módulo, então dizemos que $M$ é \emph{graduado} se $M$ pode ser escrito como 
$$
M=M_0\oplus M_1\oplus\cdots
$$ 
como um grupo abeliano em tal forma que $R_iM_j\subseteq M_{i+j}$ para todo $i,j\geq 0$. 


Assuma que $R=k[x_1,\ldots,x_n]/I$ com um ideal graduado $I$ e seja $M$ um $R$-módulo graduado 
finitamente gerado. Definimos a função de Hilbert  
$$
\chi_M(n)=\dim_k M_n.
$$

\begin{exercise}
    Mostre que $\chi_M(n)<\infty$ para todo $n$.
\end{exercise}



Um morfismo {\em homogêneo} de $R$-módulos graduados $M$ e $N$ é um morfismos $\varphi:M\to N$ 
que satisfaz a condição $\varphi(M_i)\subseteq N_i$. 

\begin{theorem} Seja $R=k[x_1,\ldots,x_n]/I$ onde $I$ é um ideal homogêneo (e assim $R$ é graduado). 
Seja $M$ um $R$-módulo graduado. Então existe um polinômio $p_M(t)\in\mathbb Q[t]$ de grau menor ou igual a 
$n-1$ tal que $\chi_M(m)=\dim M_m=p_M(m)$ para todo $m\in\mathbb N$ suficientemente grande.  
\end{theorem}


\begin{proof} 
    Indução por $n$. Se $n=0$, então $R=k$ e temos que $M$ é um $k$-espaço vetorial de dimensão finita e podemos tomar $p_M(t)=0$. 

Assuma que $n\geq 1$ e o teorema está verdadeiro para $n-1$. Seja $M[1]$ o $R$-módulo graduado 
tal que $M[1]=M$ como $R$-módulos e $M[1]_i=M_{i+1}$ para todo $i$. É fácil verificar que $M[1]$ é um 
$R$-módulo graduado. Seja $\mu:M\to M[1]$ dado por $m\mapsto x_n\cdot m$ (multiplicação por $x_n$).
Então $\mu$ é um morfismo homogêneo de $R$-módulos graduados. 
Temos a seguinte sequência exata de $R$-módulos
$$
    0\to N \to M\to M[1]\to P\to 0
$$
onde $N$ e $P$ são o núcleo e conúcleo de $\mu$. Note que $N$ e $P$ são finitamente gerados sobre $R$, 
pois $R$ é noetheriano. Além disso, $N$, $M[1]$ e $P$ são graduados, o morfismo $\mu$ é homogêneo, e temos as seguintes sequências 
exatas de $k$-espaços 
para todo $m\geq 0$: 
$$
    0\to N_m \to M_m\to (M[1])_m\to P_m\to 0.
$$
Olhando nas dimensões obtemos  
$$
\dim_k P_m=\dim_k N_m-\dim_k M_m+\dim_k M_{m+1};
$$
ou seja
$$
\chi_M(m+1)-\chi_M(m)=\chi_P(m)-\chi_N(m).
$$
Considere $R_1=R/(x_r)\cong R/(I+(x_r))$. Então $x_r$ anula $P$ e $N$ e podemos considerar $P$ e $N$ como 
$R_1$-módulos finitamente gerados. Assim $\chi_N(t)$ e $\chi_P(t)$ são funções polinomiais de grau menor 
ou igaul a $n-2$. Assim $\chi_M(t)$ é polinomial de grau menor ou igual a $n-1$. 
\end{proof}


Seja $(R,\m,k)$ um anel local noetheriano. Para $n\geq 0$, definimos 
\[
    \lambda_R(n)=\len{A/\m^n}
\]
onde $\m^0=A$. 



\begin{theorem}
Assuma que $(R,\m,k)$ é um anel local noetheriano. Então 
\[
    \Delta\lambda_R(n)=\dim_k\m^n/\m^{n+1}.
\]
Existe um polinômio $p(t)\in\Q[t]$ tal que $\lambda_A(n)=p(n)$ para todo $n\in\N$ suficientemente grande. 
\end{theorem}
\begin{proof}
    Primeiro 
    \[
        \len A/\m^{n+1}=\len A/\m+\len \m/\m^2+\cdots+\len \m^n/\m^{n+1}
    \]
e
\[
    \Delta\lambda_R(n)=\lambda_R(n+1)-\lambda_R(n)=\len \m^n/\m^{n+1}.
\]
Como $\m$ anula $\m^i/\m^{i+1}$ para todo $i$, temos que $\m^i/\m^{i+1}$ pode ser visto como um $k$-espaço e 
os $R$-submódulos de $\m^i/\m^{i+1}$ são precisamente os $k$-subespaços. 
Assim $\len \m^{i}/\m^{i+1}=\dim_k\m^i/\m^{i+1}$. 
\end{proof}
O polinômio $p(n)$ no teorema anteirior chama-se o polinômio de {\em Hilbert--Samuel} do anel local $R$.
\end{document}
