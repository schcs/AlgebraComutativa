\documentclass[12pt]{amsart}
\usepackage{amsthm}
\usepackage{amssymb}
\usepackage{showkeys}
\usepackage{tikz-cd}

\renewcommand{\a}{\mathfrak a}
\renewcommand{\b}{\mathfrak b}
\newcommand{\m}{\mathfrak m}
\newcommand{\n}{\mathfrak n}
\newcommand{\p}{\mathfrak p}
\newcommand{\q}{\mathfrak q}
\renewcommand{\r}{\mathfrak r}
\newcommand{\F}{\mathbb F}
\renewcommand{\L}{\mathbb L}
\newcommand{\Q}{\mathbb Q}
\newcommand{\N}{\mathbb N}
\newcommand{\Z}{\mathbb Z}


\newcommand{\K}{\mathbb K}
\newcommand{\fracf}[1]{\mbox{Frac}(#1)}
\newcommand{\spec}[1]{\mbox{Spec}(#1)}
\newcommand{\len}{\mbox{len}\,}

\newtheorem{theorem}{Theorem}
\newtheorem{corollary}{Corollary}[theorem]
\newtheorem{lemma}[theorem]{Lemma}
\newtheorem{exercise}[theorem]{Exercício}

\theoremstyle{definition}
\newtheorem{example}[theorem]{Exemplo}


\oddsidemargin 0pt
\evensidemargin 0pt
\textheight 8.1in \textwidth 6.3in


\relpenalty=10000
\binoppenalty=10000
\tolerance=500


\begin{document}

\Large
\title{O Teorema de Artin-Rees}
\maketitle

Seja $M$ um $R$-módulo e $\a\subseteq R$ um ideal. Considere os conjuntos $m+\a^iM$ com $m\in M$ e 
$i\geq 0$. Estes conjuntos formam uma base aberta para uma topologia sobre $M$ chamado de 
\emph{topologia $\a$-ádica} e $M$ pode ser considerado como um espaço topológico. 

\begin{lemma}
    A adição e o multiplicação com escalar são funções contínuas sobre $M$ na $\a$-ésima topologia. 
    Além disso $M$ é Haussdorf na $\a$-ésima topologia se s somente se $\bigcap_{i\geq 0}\a^iM=0$.
\end{lemma}
\begin{proof} Exercício.
\end{proof}
Um caso particular é quando $M=R$ e neste caso temos que $x+\a^i$ com $x\in R$ e $i\geq 0$ formam uma 
base para a topologia sobre $R$. 

Se $N$ é um submódulo de $M$, temos que $N$ pode ser visto com a própria $\a$-ésima topologia e também com 
a topologia induzida da $\a$-ésima topologia de $M$. O seguinte resultado implica que as duas topologias são as mesmas.

\begin{theorem}[Artin--Rees]
Sejam $R$ um anel noetheriano, $\a\subseteq R$ ideal e $M$ um $R$-módulo finitamente gerado. Seja $N\subseteq M$
um submódulo. Então existe $r\in\N$ tal que 
\[
    (\a^nM)\cap N=\a^{n-r}(\a^r M\cap N)
\]
para todo $n\geq r$. Em particular, para todo $n$ suficientemente grande
\[
\a^nN\subseteq (\a^nM)\cap N\subseteq \a^{n-r}N.
\]
\end{theorem}
\begin{proof}
    Assuma que $\a=(x_1,\ldots,x_u)_R$ e $M=(m_1,\ldots,m_v)$. 
    Considere a álgebra $B_\a(R)$ e o módulo $B_\a(M)$ "blow-up"definidos como 
\[
    B_\a(R)=\bigoplus_{n\geq 0}\a^n\quad\mbox{e}\quad
    B_\a(M)=\bigoplus_{n\geq 0}\a^nM.
\]
A álgebra $B_\a(R)$ é finitamente gerada como $R$-álgebra, pois ela é gerada pelos 
elementos $(0,x_i,0,\ldots)$ para $i\in\{1,\ldots,u\}$. Portanto $B_\a(R)$ é notheriana.
Além disso, $B_\a(M)$ é gerado por $(m_i,0.\ldots,)$ com $i\in\{1,\ldots,v\}$ e $B_\a(M)$ é finitamente
gerado como $B_\a(R)$-módulo. 

A inclusão $\a^{n-r}(\a^rM\cap N)\subseteq \a^nM\cap N$ é clara. Para 
a inclusão oposta, considere 
\[
    P=\bigoplus_{n\geq 0}\a^nM\cap N
\]
como um submódulo de $B_\a(M)$. Como $B_\a(M)$ é finitamente gerado sobre a álgebra noetheriana $B_\a(R)$, $P$ é finitamente gerado como $B_\a(R)$-módulo. Seja $r$ o grau máximo de 
um elemento em um conjunto de geradores homogêneos. Então para $n\geq r$, temos que 
\[
    P_n=\sum_{i=0}^r B_\a(R)_{n-i}P_i
\]
logo 
\[
    (\a^n M)\cap N=\sum_{i=0}^r \a^{n-i}((\a^iM)\cap N)\subseteq \a^{n-r}((\a^rM)\cap N).
\]
A  última igualdade segue, pois um elemento de $\a^{n-i}((\a^iM)\cap N)$ tem a forma 
\[
    x_1\cdots x_{n-i}(x_{n-i+1}\cdots x_n)m
\]
com $x_i\in\a$ e $m\in M$ e $(x_{n-i+1}\cdots x_n)m\in N$. Se $r\geq i$, então o mesmo elemento pode ser escrito como 
\[
    x_1\cdots x_{n-r}(x_{n-r+1}\cdots x_{n-i+1}\cdots x_n)m\in \a^{n-r}((\a^r M)\cap N).
\]
\end{proof}
\end{document}
