\documentclass[12pt]{amsart}
\usepackage{amsthm}
\usepackage{amssymb}
\usepackage{showkeys}
\usepackage{tikz-cd}

\renewcommand{\a}{\mathfrak a}
\renewcommand{\b}{\mathfrak b}
\newcommand{\m}{\mathfrak m}
\newcommand{\n}{\mathfrak n}
\newcommand{\p}{\mathfrak p}
\newcommand{\q}{\mathfrak q}
\renewcommand{\r}{\mathfrak r}
\newcommand{\F}{\mathbb F}
\renewcommand{\L}{\mathbb L}
\newcommand{\K}{\mathbb K}
\newcommand{\fracf}[1]{\mbox{Frac}(#1)}
\newcommand{\spec}[1]{\mbox{Spec}(#1)}


\newtheorem{theorem}{Theorem}
\newtheorem{corollary}{Corollary}[theorem]
\newtheorem{lemma}[theorem]{Lemma}
\newtheorem{exercise}[theorem]{Exercício}
\newtheorem{definition}[theorem]{Definition}
\newtheorem{example}[theorem]{Example}


\oddsidemargin 0pt
\evensidemargin 0pt
\textheight 8.1in \textwidth 6.3in


\relpenalty=10000
\binoppenalty=10000
\tolerance=500


\begin{document}

\Large
\title{Dimension}
\maketitle

\begin{lemma}
    Assuma que $k$ é um corpo e $R$ é um $\K$-domínio finitamente gerado. Assuma que 
    $\p_1\subset\p_2\subset\cdots\subset \p_r$ é uma cadeia em $\spec R$ e seja $d$ o grau de transcendência de $\fracf R$ sobre $k$. Então $r\leq d$ com igualdade se e somente se a cadeia $(\p_i)$ é saturada no sentido que ela não é refinamento de uma cadeia mais longa. 
\end{lemma}
\begin{proof}
Use o Teorema de Normalização de Noether para escolher $t_1,\ldots,t_n\in R$ algebricamente independentes tal que 
\[
    \p_i\cap P=(t_1,\ldots,t_{h_i})_P
\] 
para $i\in\{1,\ldots,r\}$ onde $P=k[t_1,\ldots,t_n]$. 
Ponha $\L=\fracf R$ e $\K=\fracf P$. Se $a/b\in\L$, então $a$ e $b$ são 
integrais sobre $R$ e $a$ e $b$ são obviamente algébricos sobre $\K=\fracf R$. Assim $a/b$ é algébrico sobre 
$\K$. Logo, a extensão $\K\subseteq \L$ é algébrica e assim 
$d=n$. O primo $\p_i$ está acima de $\p_i\cap P$ e o lema de incomparabilidade implica que 
$\p_i\cap P\neq \p_{i+1}\cap P$ para todo $i$ e $h_1<h_2<\cdots<h_r$. Logo $r\leq n=d$. 
Em particular, $d$ é o maior valor possível para $r$ e neste caso a cadeia é saturada. 

Assuma que a cadeia $(\p_i)$ é saturada e tem comrimento $r$. Neste caso $\p_1=0$ e $h_1=0$ e $\p_r$ é maximal. 
Como $P\subseteq R$ é integral, temos que $\p_r\cap P$ é maximal e $h_r=n$. 
Se $r<n$, precisa existir algum $i$ tal que $h_{i}+2\leq h_{i+1}$ temos que 
\[
    \p_i\cap P=(t_1,\ldots,t_{h_i})_P\subset (t_1,\ldots,t_{h_{i}},t_{h_i+1}) \subset 
    (t_1,\ldots,t_{h_{i+1}})_P=\p_{i+1}\cap P.
\] 
Ponha $\r= (t_1,\ldots,t_{h_{i}},t_{h_i+1})_P$. 
Então $P/(\p_i\cap P)\cong k[t_{h_i+1},\ldots,t_n]$ é um anel de polinômios sobre $k$, e 
é um domínio normal. Além disso, o mapa $P/(\p_i\cap P)\to R/\p_i$ definido por $p+(\p_i\cap P)\mapsto p+\p_i$ é injetiva e $P/(\p_i\cap P)\subseteq R/\p_i$
é uma extensão integral, pois $P\subseteq R$ é integral. Pelo Teorema de Going-down, existe 
um primo $\p\in \spec R$ tal que $\p/\p_i\in\spec {R/\p_i}$  está acima de $\r/(\p_i\cap P)$ e 
$\p/\p_i\subset \p_{i+1}/\p_i$. Como $\r/(\p_i\cap P)\neq 0$, $\p\neq \p_i$ e  $\p_i\subset \p\subset \p_{i+1}$. Obtivemos uma contradição, pois a cadeia 
$(\p_i)$ foi assumida saturada.   
\end{proof}

%\begin{definition}
    Seja $R$ um anel. A dimensão de Krull de $R$ é o supremo dos comprimentos das cadeias 
    \[
        \p_1\subset\p_2\subset \cdots \subset \p_k
    \]
    de ideais primos em $R$. A dimensão de Krull está denotado por $\dim R$. 
%\end{definition}

\begin{example}
    A dimensão de um corpo é zero.
\end{example}

\begin{theorem}
    Se $R$ é um $k$-domínio finitamente gerado, então $\dim R$ é igual ao grau de transcendência de $\fracf R$ sobre $k$. 
\end{theorem}

\begin{theorem}
    Seja $R$ um $k$-domínio finitamente gerado, seja $\p\in\spec R$ e seja $\m$ um ideal maximal em $R$.   
    Então 
    \[
        \dim R_{(\p)}+\dim R/\p=\dim R\quad\mbox{e}\quad \dim R_{(\m)}=\dim R. 
    \]
\end{theorem}
\begin{proof}
Assuma que 
\[
    \p_1\subset \p_2\subset \cdots\subset \p=\p_i\subset\cdots \subset \p_r
\]
é uma cadeia saturada dos primos que contém $\p$. Então $r$ é   igual ao grau de transcendência de $R$ sobre $k$ e $r=\dim R$. Por outro lado 
\begin{equation}\label{eq:c1}
    \p_i/\p\subset \cdots\subset \p_r/\p_i
\end{equation}
é uma cadeia em $\spec{R/\p}$ e 
\begin{equation}\label{eq:c2}
    \p_1 R_{(\p)}\subset \cdots \subset\p_i R_{(\p)}
\end{equation}
é uma cadeia em $\spec{R_{(\p)}}$. Assim $\dim R/\p+\dim R_{(\p)}\geq \dim R\geq r$. 
Por outro lado, cadeias como \eqref{eq:c1}--\eqref{eq:c2} resultam em cadeias em $\spec R$ e assim $\dim R\geq 
\dim R/\p+\dim R_{(\p)}$.  Logo  
\[
    \dim R_{(\p)}+\dim R/\p=\dim R.
\] 
A segunda afirmação segue da primeira observando que $\dim R/\m=0$ como $R/\m$ é um corpo.
\end{proof}
\end{document}
